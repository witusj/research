% Options for packages loaded elsewhere
\PassOptionsToPackage{unicode}{hyperref}
\PassOptionsToPackage{hyphens}{url}
%


\PassOptionsToPackage{table}{xcolor}

\documentclass[
  10pt,
  letterpaper,
]{article}

\usepackage{amsmath,amssymb}
\usepackage{lmodern}
\usepackage{iftex}
\ifPDFTeX
  \usepackage[T1]{fontenc}
  \usepackage[utf8]{inputenc}
  \usepackage{textcomp} % provide euro and other symbols
\else % if luatex or xetex
  \usepackage{unicode-math}
  \defaultfontfeatures{Scale=MatchLowercase}
  \defaultfontfeatures[\rmfamily]{Ligatures=TeX,Scale=1}
\fi
% Use upquote if available, for straight quotes in verbatim environments
\IfFileExists{upquote.sty}{\usepackage{upquote}}{}
\IfFileExists{microtype.sty}{% use microtype if available
  \usepackage[]{microtype}
  \UseMicrotypeSet[protrusion]{basicmath} % disable protrusion for tt fonts
}{}
\makeatletter
\@ifundefined{KOMAClassName}{% if non-KOMA class
  \IfFileExists{parskip.sty}{%
    \usepackage{parskip}
  }{% else
    \setlength{\parindent}{0pt}
    \setlength{\parskip}{6pt plus 2pt minus 1pt}}
}{% if KOMA class
  \KOMAoptions{parskip=half}}
\makeatother
\usepackage{xcolor}
\usepackage[top=0.85in,left=2.75in,footskip=0.75in]{geometry}
\setlength{\emergencystretch}{3em} % prevent overfull lines
\setcounter{secnumdepth}{-\maxdimen} % remove section numbering

\usepackage{color}
\usepackage{fancyvrb}
\newcommand{\VerbBar}{|}
\newcommand{\VERB}{\Verb[commandchars=\\\{\}]}
\DefineVerbatimEnvironment{Highlighting}{Verbatim}{commandchars=\\\{\}}
% Add ',fontsize=\small' for more characters per line
\usepackage{framed}
\definecolor{shadecolor}{RGB}{241,243,245}
\newenvironment{Shaded}{\begin{snugshade}}{\end{snugshade}}
\newcommand{\AlertTok}[1]{\textcolor[rgb]{0.68,0.00,0.00}{#1}}
\newcommand{\AnnotationTok}[1]{\textcolor[rgb]{0.37,0.37,0.37}{#1}}
\newcommand{\AttributeTok}[1]{\textcolor[rgb]{0.40,0.45,0.13}{#1}}
\newcommand{\BaseNTok}[1]{\textcolor[rgb]{0.68,0.00,0.00}{#1}}
\newcommand{\BuiltInTok}[1]{\textcolor[rgb]{0.00,0.23,0.31}{#1}}
\newcommand{\CharTok}[1]{\textcolor[rgb]{0.13,0.47,0.30}{#1}}
\newcommand{\CommentTok}[1]{\textcolor[rgb]{0.37,0.37,0.37}{#1}}
\newcommand{\CommentVarTok}[1]{\textcolor[rgb]{0.37,0.37,0.37}{\textit{#1}}}
\newcommand{\ConstantTok}[1]{\textcolor[rgb]{0.56,0.35,0.01}{#1}}
\newcommand{\ControlFlowTok}[1]{\textcolor[rgb]{0.00,0.23,0.31}{#1}}
\newcommand{\DataTypeTok}[1]{\textcolor[rgb]{0.68,0.00,0.00}{#1}}
\newcommand{\DecValTok}[1]{\textcolor[rgb]{0.68,0.00,0.00}{#1}}
\newcommand{\DocumentationTok}[1]{\textcolor[rgb]{0.37,0.37,0.37}{\textit{#1}}}
\newcommand{\ErrorTok}[1]{\textcolor[rgb]{0.68,0.00,0.00}{#1}}
\newcommand{\ExtensionTok}[1]{\textcolor[rgb]{0.00,0.23,0.31}{#1}}
\newcommand{\FloatTok}[1]{\textcolor[rgb]{0.68,0.00,0.00}{#1}}
\newcommand{\FunctionTok}[1]{\textcolor[rgb]{0.28,0.35,0.67}{#1}}
\newcommand{\ImportTok}[1]{\textcolor[rgb]{0.00,0.46,0.62}{#1}}
\newcommand{\InformationTok}[1]{\textcolor[rgb]{0.37,0.37,0.37}{#1}}
\newcommand{\KeywordTok}[1]{\textcolor[rgb]{0.00,0.23,0.31}{#1}}
\newcommand{\NormalTok}[1]{\textcolor[rgb]{0.00,0.23,0.31}{#1}}
\newcommand{\OperatorTok}[1]{\textcolor[rgb]{0.37,0.37,0.37}{#1}}
\newcommand{\OtherTok}[1]{\textcolor[rgb]{0.00,0.23,0.31}{#1}}
\newcommand{\PreprocessorTok}[1]{\textcolor[rgb]{0.68,0.00,0.00}{#1}}
\newcommand{\RegionMarkerTok}[1]{\textcolor[rgb]{0.00,0.23,0.31}{#1}}
\newcommand{\SpecialCharTok}[1]{\textcolor[rgb]{0.37,0.37,0.37}{#1}}
\newcommand{\SpecialStringTok}[1]{\textcolor[rgb]{0.13,0.47,0.30}{#1}}
\newcommand{\StringTok}[1]{\textcolor[rgb]{0.13,0.47,0.30}{#1}}
\newcommand{\VariableTok}[1]{\textcolor[rgb]{0.07,0.07,0.07}{#1}}
\newcommand{\VerbatimStringTok}[1]{\textcolor[rgb]{0.13,0.47,0.30}{#1}}
\newcommand{\WarningTok}[1]{\textcolor[rgb]{0.37,0.37,0.37}{\textit{#1}}}

\providecommand{\tightlist}{%
  \setlength{\itemsep}{0pt}\setlength{\parskip}{0pt}}\usepackage{longtable,booktabs,array}
\usepackage{calc} % for calculating minipage widths
% Correct order of tables after \paragraph or \subparagraph
\usepackage{etoolbox}
\makeatletter
\patchcmd\longtable{\par}{\if@noskipsec\mbox{}\fi\par}{}{}
\makeatother
% Allow footnotes in longtable head/foot
\IfFileExists{footnotehyper.sty}{\usepackage{footnotehyper}}{\usepackage{footnote}}
\makesavenoteenv{longtable}
\usepackage{graphicx}
\makeatletter
\def\maxwidth{\ifdim\Gin@nat@width>\linewidth\linewidth\else\Gin@nat@width\fi}
\def\maxheight{\ifdim\Gin@nat@height>\textheight\textheight\else\Gin@nat@height\fi}
\makeatother
% Scale images if necessary, so that they will not overflow the page
% margins by default, and it is still possible to overwrite the defaults
% using explicit options in \includegraphics[width, height, ...]{}
\setkeys{Gin}{width=\maxwidth,height=\maxheight,keepaspectratio}
% Set default figure placement to htbp
\makeatletter
\def\fps@figure{htbp}
\makeatother

% Use adjustwidth environment to exceed column width (see example table in text)
\usepackage{changepage}

% marvosym package for additional characters
\usepackage{marvosym}

% cite package, to clean up citations in the main text. Do not remove.
% Using natbib instead
% \usepackage{cite}

% Use nameref to cite supporting information files (see Supporting Information section for more info)
\usepackage{nameref,hyperref}

% line numbers
\usepackage[right]{lineno}

% ligatures disabled
\usepackage{microtype}
\DisableLigatures[f]{encoding = *, family = * }

% create "+" rule type for thick vertical lines
\newcolumntype{+}{!{\vrule width 2pt}}

% create \thickcline for thick horizontal lines of variable length
\newlength\savedwidth
\newcommand\thickcline[1]{%
  \noalign{\global\savedwidth\arrayrulewidth\global\arrayrulewidth 2pt}%
  \cline{#1}%
  \noalign{\vskip\arrayrulewidth}%
  \noalign{\global\arrayrulewidth\savedwidth}%
}

% \thickhline command for thick horizontal lines that span the table
\newcommand\thickhline{\noalign{\global\savedwidth\arrayrulewidth\global\arrayrulewidth 2pt}%
\hline
\noalign{\global\arrayrulewidth\savedwidth}}

% Text layout
\raggedright
\setlength{\parindent}{0.5cm}
\textwidth 5.25in 
\textheight 8.75in

% Bold the 'Figure #' in the caption and separate it from the title/caption with a period
% Captions will be left justified
\usepackage[aboveskip=1pt,labelfont=bf,labelsep=period,justification=raggedright,singlelinecheck=off]{caption}
\renewcommand{\figurename}{Fig}

% Remove brackets from numbering in List of References
\makeatletter
\renewcommand{\@biblabel}[1]{\quad#1.}
\makeatother

% Header and Footer with logo
\usepackage{lastpage,fancyhdr}
\usepackage{epstopdf}
%\pagestyle{myheadings}
\pagestyle{fancy}
\fancyhf{}
%\setlength{\headheight}{27.023pt}
%\lhead{\includegraphics[width=2.0in]{PLOS-submission.eps}}
\rfoot{\thepage/\pageref{LastPage}}
\renewcommand{\headrulewidth}{0pt}
\renewcommand{\footrule}{\hrule height 2pt \vspace{2mm}}
\fancyheadoffset[L]{2.25in}
\fancyfootoffset[L]{2.25in}
\lfoot{\today}
\makeatletter
\makeatother
\makeatletter
\makeatother
\makeatletter
\@ifpackageloaded{caption}{}{\usepackage{caption}}
\AtBeginDocument{%
\ifdefined\contentsname
  \renewcommand*\contentsname{Table of contents}
\else
  \newcommand\contentsname{Table of contents}
\fi
\ifdefined\listfigurename
  \renewcommand*\listfigurename{List of Figures}
\else
  \newcommand\listfigurename{List of Figures}
\fi
\ifdefined\listtablename
  \renewcommand*\listtablename{List of Tables}
\else
  \newcommand\listtablename{List of Tables}
\fi
\ifdefined\figurename
  \renewcommand*\figurename{Figure}
\else
  \newcommand\figurename{Figure}
\fi
\ifdefined\tablename
  \renewcommand*\tablename{Table}
\else
  \newcommand\tablename{Table}
\fi
}
\@ifpackageloaded{float}{}{\usepackage{float}}
\floatstyle{ruled}
\@ifundefined{c@chapter}{\newfloat{codelisting}{h}{lop}}{\newfloat{codelisting}{h}{lop}[chapter]}
\floatname{codelisting}{Listing}
\newcommand*\listoflistings{\listof{codelisting}{List of Listings}}
\makeatother
\makeatletter
\@ifpackageloaded{caption}{}{\usepackage{caption}}
\@ifpackageloaded{subcaption}{}{\usepackage{subcaption}}
\makeatother
\makeatletter
\makeatother
\ifLuaTeX
  \usepackage{selnolig}  % disable illegal ligatures
\fi
\usepackage[numbers,square,comma]{natbib}
\bibliographystyle{plos2015}
\IfFileExists{bookmark.sty}{\usepackage{bookmark}}{\usepackage{hyperref}}
\IfFileExists{xurl.sty}{\usepackage{xurl}}{} % add URL line breaks if available
\urlstyle{same} % disable monospaced font for URLs
\hypersetup{
  pdftitle={PLOS template},
  pdfauthor={Name1 Surname; Name2 Surname; Name3 Surname; Name4 Surname; Name5 Surname; Name6 Surname; Name7 Surname; with the Lorem Ipsum Consortium},
  hidelinks,
  pdfcreator={LaTeX via pandoc}}



\begin{document}
\vspace*{0.2in}

% Title must be 250 characters or less.
\begin{flushleft}
{\Large
\textbf\newline{PLOS
template} % Please use "sentence case" for title and headings (capitalize only the first word in a title (or heading), the first word in a subtitle (or subheading), and any proper nouns).
}
\newline
\\
% Insert author names, affiliations and corresponding author email (do not include titles, positions, or degrees).
Name1 Surname\textsuperscript{1,2\Yinyang}, Name2
Surname\textsuperscript{2\Yinyang}, Name3
Surname\textsuperscript{2,3}, Name4 Surname\textsuperscript{2}, Name5
Surname\textsuperscript{2\dag}, Name6 Surname\textsuperscript{2}, Name7
Surname\textsuperscript{1,2,3*}, with the Lorem Ipsum
Consortium\textsuperscript{\textpilcrow}
\\
\bigskip
\textbf{1} Affiliation Dept, Institution
Name, City, State, Country, \\ \textbf{2} Affiliation Center,
Institution Name, City, State, Country, \\ \textbf{3} Affiliation
Program, Institution Name, City, State, Country, 
\bigskip

% Insert additional author notes using the symbols described below. Insert symbol callouts after author names as necessary.
% 
% Remove or comment out the author notes below if they aren't used.
%
% Primary Equal Contribution Note
\Yinyang These authors contributed equally to this work.

% Additional Equal Contribution Note
% Also use this double-dagger symbol for special authorship notes, such as senior authorship.
%\ddag These authors also contributed equally to this work.

% Current address notes
\textcurrency Current Address: Dept/Program/Center, Institution Name, City, State, Country % change symbol to "\textcurrency a" if more than one current address note
% \textcurrency b Insert second current address 
% \textcurrency c Insert third current address

% Deceased author note
\dag Deceased

% Group/Consortium Author Note
\textpilcrow Membership list can be found in the Acknowledgments
sections

% Use the asterisk to denote corresponding authorship and provide email address in note below.
* correspondingauthor@institute.edu

\end{flushleft}

\section*{Abstract}
Lorem ipsum dolor sit amet, consectetur adipiscing elit. Curabitur eget
porta erat. Morbi consectetur est vel gravida pretium. Suspendisse ut
dui eu ante cursus gravida non sed sem. Nullam sapien tellus, commodo id
velit id, eleifend volutpat quam. Phasellus mauris velit, dapibus
finibus elementum vel, pulvinar non tellus. Nunc pellentesque pretium
diam, quis maximus dolor faucibus id. Nunc convallis sodales ante, ut
ullamcorper est egestas vitae. Nam sit amet enim ultrices, ultrices elit
pulvinar, volutpat risus.

\section*{Author summary}
Lorem ipsum dolor sit amet, consectetur adipiscing elit. Curabitur eget
porta erat. Morbi consectetur est vel gravida pretium. Suspendisse ut
dui eu ante cursus gravida non sed sem. Nullam sapien tellus, commodo id
velit id, eleifend volutpat quam. Phasellus mauris velit, dapibus
finibus elementum vel, pulvinar non tellus. Nunc pellentesque pretium
diam, quis maximus dolor faucibus id. Nunc convallis sodales ante, ut
ullamcorper est egestas vitae. Nam sit amet enim ultrices, ultrices elit
pulvinar, volutpat risus.

\linenumbers\hypertarget{general-guidelines-for-this-quarto-template}{%
\section{General guidelines for this Quarto
template}\label{general-guidelines-for-this-quarto-template}}

This template shows how to use PLOS template from
\url{https://plos.org/resources/writing-center/}. Each journal have a
submission guideline page, please refer to it.

\begin{itemize}
\tightlist
\item
  \href{https://journals.plos.org/plosbiology/s/submission-guidelines}{PLOS
  Biology}
\item
  \href{https://journals.plos.org/climate/s/submission-guidelines}{PLOS
  Climate}
\item
  \href{https://journals.plos.org/digitalhealth/s/submission-guidelines}{PLOS
  Digital Health}
\item
  \href{https://journals.plos.org/ploscompbiol/s/submission-guidelines}{PLOS
  Computational Biology}
\item
  \href{https://journals.plos.org/plosgenetics/s/submission-guidelines}{PLOS
  Genetics}
\item
  \href{https://journals.plos.org/globalpublichealth/s/submission-guidelines}{PLOS
  Global Public Health}
\item
  \href{https://journals.plos.org/plosmedicine/s/submission-guidelines}{PLOS
  Medicine}
\item
  \href{https://journals.plos.org/plosntds/s/submission-guidelines}{PLOS
  Neglected Tropical Diseases}
\item
  \href{https://journals.plos.org/plosone/s/submission-guidelines}{PLOS
  ONE}
\item
  \href{https://journals.plos.org/plospathogens/s/submission-guidelines}{PLOS
  Pathogens}
\item
  \href{https://journals.plos.org/sustainabilitytransformation/s/submission-guidelines}{PLOS
  Sustainability and Transformation}
\item
  \href{https://journals.plos.org/water/s/submission-guidelines}{PLOS
  Water}
\end{itemize}

This template file contains some guidelines and recommandation initially
given in \texttt{plos\_latex\_template.tex} that can be found in
\url{https://github.com/quarto-journals/plos/blob/main/style-guide/plos_latex_template.tex}

\hypertarget{metadata}{%
\subsection{Metadata}\label{metadata}}

\hypertarget{about-journal-id-field}{%
\subsubsection{About journal id field}\label{about-journal-id-field}}

This is an identifier for the target journal. It can be derived from
https://plos.org/resources/writing-center/ following submission
guidelines link, the identifier is the part of the URL after
\texttt{https://journals.plos.org/\textless{}id\textgreater{}/s/submission-guidelines}

\begin{longtable}[]{@{}ll@{}}
\toprule()
Journal & id \\
\midrule()
\endhead
PLOS Biology & plosbiology \\
PLOS Climate & climate \\
PLOS Digital Health & digitalhealth \\
PLOS Computational Biology & ploscompbiol \\
PLOS Genetics & plosgenetics \\
PLOS Global Public Health & globalpublichealth \\
PLOS Medicine & plosmedicine \\
PLOS Neglected Tropical Diseases & plosntds \\
PLOS ONE & plosone \\
PLOS Pathogens & plospathogens \\
PLOS Sustainability and Transformation & sustainabilitytransformation \\
PLOS Water & water \\
\bottomrule()
\end{longtable}

Example :

\begin{Shaded}
\begin{Highlighting}[]
\FunctionTok{format}\KeywordTok{:}
\AttributeTok{  }\FunctionTok{plos{-}pdf}\KeywordTok{:}
\AttributeTok{    }\FunctionTok{journal}\KeywordTok{:}
\AttributeTok{      }\FunctionTok{id}\KeywordTok{:}\AttributeTok{ water}
\end{Highlighting}
\end{Shaded}

\hypertarget{once-your-paper-is-accepted-for-publication}{%
\subsection{Once your paper is accepted for
publication,}\label{once-your-paper-is-accepted-for-publication}}

Do not include track change in LaTeX file and leave only the final text
of your manuscript. PLOS recommends the use of latexdiff to track
changes during review, as this will help to maintain a clean tex file.
Visit \url{https://www.ctan.org/pkg/latexdiff?lang=en} for info or
contact us at \href{mailto:latex@plos.org}{\nolinkurl{latex@plos.org}}.

\emph{This should not be a problem using Quarto but still a
recommandation from the journal}

There are no restrictions on package use within the LaTeX files except
that no packages listed in the template may be deleted.

Please do not include colors or graphics in the text. Color can be used
to apply background shading to table cells only.

The manuscript LaTeX source should be contained within a single file (do
not use\texttt{\textbackslash{}input},
\texttt{\textbackslash{}externaldocument}, or similar commands).

Please contact \href{mailto:latex@plos.org}{\nolinkurl{latex@plos.org}}
with any questions submission guidelines. For anything Quarto related,
please open an issue in \url{https://github.com/quarto-journals/plos}.
If this is related to the LaTeX template, this could also be a good idea
to contact PLOS directly.

\hypertarget{figures-and-tables}{%
\subsection{Figures and Tables}\label{figures-and-tables}}

Please include tables/figure captions directly after the paragraph where
they are first cited in the text.

\hypertarget{figures}{%
\subsubsection{Figures}\label{figures}}

However, do not include graphics in your manuscript

\begin{itemize}
\tightlist
\item
  Figures should be uploaded separately from your manuscript file.
\item
  Figures generated using LaTeX should be extracted and removed from the
  PDF before submission.
\item
  Figures containing multiple panels/subfigures must be combined into
  one image file before submission.
\end{itemize}

\textbf{This means that, depending on how you create your figure, a
manual post processing will be required.}

For figure citations, please use ``Fig'' instead of ``Figure''. This has
been made the default in this Quarto format:

\begin{Shaded}
\begin{Highlighting}[]
\FunctionTok{crossref}\KeywordTok{:}
\AttributeTok{  }\FunctionTok{fig{-}title}\KeywordTok{:}\AttributeTok{ Fig }
\end{Highlighting}
\end{Shaded}

Also, place figure captions after the first paragraph in which they are
cited.

See PLOS figure guidelines at
\url{https://journals.plos.org/plosone/s/figures} and in your specific
journal guideline.

\hypertarget{tables}{%
\subsubsection{Tables}\label{tables}}

Tables should be cell-based and may not contain:

\begin{itemize}
\tightlist
\item
  spacing/line breaks within cells to alter layout or alignment
\item
  do not nest tabular environments (no tabular environments within
  tabular environments)
\item
  no graphics or colored text (cell background color/shading OK)
\end{itemize}

See PLOS table guidelines at
\url{http://journals.plos.org/plosone/s/tables} and in your specific
journal guideline.

For tables that exceed the width of the text column, use the adjustwidth
environment as illustrated in the example table in text below. If you
are in this case, you'll either need to manually post process the
\texttt{.tex} file and recreate the PDF, or you need to include LaTeX
tables directly.

Also, place tables after the first paragraph in which they are cited.

\hypertarget{equations-math-symbols-subscripts-and-superscripts}{%
\subsection{Equations, math symbols, subscripts, and
superscripts}\label{equations-math-symbols-subscripts-and-superscripts}}

Below are a few tips to help format your equations and other special
characters according to our specifications. For more tips to help reduce
the possibility of formatting errors during conversion, please see our
LaTeX guidelines at http://journals.plos.org/plosone/s/latex

\begin{itemize}
\item
  For inline equations, please be sure to include all portions of an
  equation in the math environment. For example, \texttt{x\$\^{}2\$} is
  incorrect; this should be formatted as \(x^2\) (or \(\mathrm{x}^2\) if
  the romanized font is desired).
\item
  Do not include text that is not math in the math environment. For
  example, \texttt{CO2} should be written as
  \texttt{CO\textbackslash{}textsubscript\{2\}} giving
  CO\textsubscript{2} instead of \texttt{CO\$\_2\$}.
\item
  Please add line breaks to long display equations when possible in
  order to fit size of the column.
\item
  For inline equations, please do not include punctuation (commas, etc)
  within the math environment unless this is part of the equation.
\item
  When adding superscript or subscripts outside of brackets/braces,
  please group using \texttt{\{\}}. For example, change
  \texttt{"{[}U(D,E,\textbackslash{}gamma){]}\^{}2"} to
  \texttt{"\{{[}U(D,E,\textbackslash{}gamma){]}\}\^{}2"}.\\
\item
  Do not use \texttt{\textbackslash{}cal} for caligraphic font. Instead,
  use \texttt{\textbackslash{}mathcal\{\}}
\end{itemize}

\hypertarget{title-and-headings}{%
\subsection{Title and headings}\label{title-and-headings}}

Please use ``sentence case'' for title and headings (capitalize only the
first word in a title (or heading), the first word in a subtitle (or
subheading), and any proper nouns).

PLOS does not support heading levels beyond the 3rd, meaning no 4th
level headings. Header 4 levels \texttt{\#\#\#\#} is used for the
\emph{Supporting information} section

\hypertarget{abstract-and-author-summary}{%
\subsection{Abstract and author
summary}\label{abstract-and-author-summary}}

Abstract must be kept below 300 words.

Author Summary must be kept between 150 and 200 words and first person
must be used.

For PLOS ONE, author summary won't be included as it is not valid for
submission.

\hypertarget{supplementary-information-syntax}{%
\subsection{Supplementary information
syntax}\label{supplementary-information-syntax}}

Use this markdown syntax to create the supplementary information block
with a custom block of class \texttt{.supp}

\begin{Shaded}
\begin{Highlighting}[]
\NormalTok{::: \{.supp\}}
\FunctionTok{\#\# SI TYPE \{\#id\}}

\NormalTok{First paragraph is a title sentence that will be bold. (required)}

\NormalTok{Optionnaly, add descriptive text after the title of the}
\NormalTok{item. No third paragraph is allowed}
\NormalTok{:::}
\end{Highlighting}
\end{Shaded}

They need to be referenced in text using \texttt{nameref} by using this
syntax \texttt{{[}id{]}(.nameref)} where \texttt{ìd} will be the id used
on the header.

\hypertarget{references}{%
\subsection{References}\label{references}}

Within Quarto, \texttt{natbib} will be used with \texttt{plos2015.bst},
which expect numeric style citation. Use brackets for references, e.g
\texttt{{[}@ref{]}}.

\hypertarget{quarto-features-limitation}{%
\subsection{Quarto features
limitation}\label{quarto-features-limitation}}

Some features are not working with this format in PDF:

\begin{itemize}
\tightlist
\item
  Callouts
\item
  Code highlighting customization (border left, background color)
\end{itemize}

\begin{center}\rule{0.5\linewidth}{0.5pt}\end{center}

\begin{quote}
\textbf{Following content of this document is from the LaTeX template
content to demo journal style.}
\end{quote}

\begin{center}\rule{0.5\linewidth}{0.5pt}\end{center}

\hypertarget{introduction}{%
\section{Introduction}\label{introduction}}

Lorem ipsum dolor sit~\citep{bib1} amet, consectetur adipiscing elit.
Curabitur eget porta erat. Morbi consectetur est vel gravida pretium.
Suspendisse ut dui eu ante cursus gravida non sed sem. Nullam
Eq~\ref{eq-schemeP} sapien tellus, commodo id velit id, eleifend
volutpat quam. Phasellus mauris velit, dapibus finibus elementum vel,
pulvinar non tellus. Nunc pellentesque pretium diam, quis maximus dolor
faucibus id.~\citep{bib2} Nunc convallis sodales ante, ut ullamcorper
est egestas vitae. Nam sit amet enim ultrices, ultrices elit pulvinar,
volutpat risus.

\begin{equation}\protect\hypertarget{eq-schemeP}{}{
\begin{aligned}
\mathrm{P_Y} = \underbrace{H(Y_n) - H(Y_n|\mathbf{V}^{Y}_{n})}_{S_Y} + \underbrace{H(Y_n|\mathbf{V}^{Y}_{n})- H(Y_n|\mathbf{V}^{X,Y}_{n})}_{T_{X\rightarrow Y}}
\end{aligned}
}\label{eq-schemeP}\end{equation}

\hypertarget{materials-and-methods}{%
\section{Materials and methods}\label{materials-and-methods}}

\hypertarget{etiam-eget-sapien-nibh}{%
\subsection{Etiam eget sapien nibh}\label{etiam-eget-sapien-nibh}}

Nulla mi mi, Fig~\ref{fig1} venenatis sed ipsum varius, volutpat euismod
diam. Proin rutrum vel massa non gravida. Quisque tempor sem et
dignissim rutrum. Lorem ipsum dolor sit amet, consectetur adipiscing
elit. Morbi at justo vitae nulla elementum commodo eu id massa. In vitae
diam ac augue semper tincidunt eu ut eros. Fusce fringilla erat
porttitor lectus cursus, vel sagittis arcu lobortis. Aliquam in enim
semper, aliquam massa id, cursus neque. Praesent faucibus semper libero.

% Place figure captions after the first paragraph in which they are cited.
\begin{figure}[!h]
\caption{{\bf Bold the figure title.}
Figure caption text here, please use this space for the figure panel descriptions instead of using subfigure commands. A: Lorem ipsum dolor sit amet. B: Consectetur adipiscing elit.}
\label{fig1}
\end{figure}

\hypertarget{results}{%
\section{Results}\label{results}}

Results and Discussion can be combined.

Nulla mi mi, venenatis sed ipsum varius, Table~\ref{table1} volutpat
euismod diam. Proin rutrum vel massa non gravida. Quisque tempor sem et
dignissim rutrum. Lorem ipsum dolor sit amet, consectetur adipiscing
elit. Morbi at justo vitae nulla elementum commodo eu id massa. In vitae
diam ac augue semper tincidunt eu ut eros. Fusce fringilla erat
porttitor lectus cursus, \nameref{s1-video} vel sagittis arcu lobortis.
Aliquam in enim semper, aliquam massa id, cursus neque. Praesent
faucibus semper libero.

% Place tables after the first paragraph in which they are cited.
\begin{table}[!ht]
\begin{adjustwidth}{-2.25in}{0in} % Comment out/remove adjustwidth environment if table fits in text column.
\centering
\caption{
{\bf Table caption Nulla mi mi, venenatis sed ipsum varius, volutpat euismod diam.}}
\begin{tabular}{|l+l|l|l|l|l|l|l|}
\hline
\multicolumn{4}{|l|}{\bf Heading1} & \multicolumn{4}{|l|}{\bf Heading2}\\ \thickhline
$cell1 row1$ & cell2 row 1 & cell3 row 1 & cell4 row 1 & cell5 row 1 & cell6 row 1 & cell7 row 1 & cell8 row 1\\ \hline
$cell1 row2$ & cell2 row 2 & cell3 row 2 & cell4 row 2 & cell5 row 2 & cell6 row 2 & cell7 row 2 & cell8 row 2\\ \hline
$cell1 row3$ & cell2 row 3 & cell3 row 3 & cell4 row 3 & cell5 row 3 & cell6 row 3 & cell7 row 3 & cell8 row 3\\ \hline
\end{tabular}
\begin{flushleft} Table notes Phasellus venenatis, tortor nec vestibulum mattis, massa tortor interdum felis, nec pellentesque metus tortor nec nisl. Ut ornare mauris tellus, vel dapibus arcu suscipit sed.
\end{flushleft}
\label{table1}
\end{adjustwidth}
\end{table}

\hypertarget{lorem-and-ipsum-nunc-blandit-a-tortor}{%
\subsection{\texorpdfstring{\textbf{LOREM}~and \textbf{IPSUM}~nunc
blandit a
tortor}{LOREM~and IPSUM~nunc blandit a tortor}}\label{lorem-and-ipsum-nunc-blandit-a-tortor}}

\hypertarget{rd-level-heading}{%
\subsubsection{3rd level heading}\label{rd-level-heading}}

Maecenas convallis mauris sit amet sem ultrices gravida. Etiam eget
sapien nibh. Sed ac ipsum eget enim egestas ullamcorper nec euismod
ligula. Curabitur fringilla pulvinar lectus consectetur pellentesque.
Quisque augue sem, tincidunt sit amet feugiat eget, ullamcorper sed
velit. Sed non aliquet felis. Lorem ipsum dolor sit amet, consectetur
adipiscing elit. Mauris commodo justo ac dui pretium imperdiet. Sed
suscipit iaculis mi at feugiat.

\begin{enumerate}
\def\labelenumi{\arabic{enumi}.}
\item
  react
\item
  diffuse free particles
\item
  increment time by dt and go to 1
\end{enumerate}

\hypertarget{sed-ac-quam-id-nisi-malesuada-congue}{%
\subsection{Sed ac quam id nisi malesuada
congue}\label{sed-ac-quam-id-nisi-malesuada-congue}}

Nulla mi mi, venenatis sed ipsum varius, volutpat euismod diam. Proin
rutrum vel massa non gravida. Quisque tempor sem et dignissim rutrum.
Lorem ipsum dolor sit amet, consectetur adipiscing elit. Morbi at justo
vitae nulla elementum commodo eu id massa. In vitae diam ac augue semper
tincidunt eu ut eros. Fusce fringilla erat porttitor lectus cursus, vel
sagittis arcu lobortis. Aliquam in enim semper, aliquam massa id, cursus
neque. Praesent faucibus semper libero.

\begin{itemize}
\item
  First bulleted item.
\item
  Second bulleted item.
\item
  Third bulleted item.
\end{itemize}

\hypertarget{discussion}{%
\section{Discussion}\label{discussion}}

Nulla mi mi, venenatis sed ipsum varius,see Table~\ref{table1} volutpat
euismod diam. Proin rutrum vel massa non gravida. Quisque tempor sem et
dignissim rutrum. Lorem ipsum dolor sit amet, consectetur adipiscing
elit. Morbi at justo vitae nulla elementum commodo eu id massa. In vitae
diam ac augue semper tincidunt eu ut eros. Fusce fringilla erat
porttitor lectus cursus, vel sagittis arcu lobortis. Aliquam in enim
semper, aliquam massa id, cursus neque. Praesent faucibus semper
libero~\citep{bib3}.

\hypertarget{conclusion}{%
\section{Conclusion}\label{conclusion}}

CO\textsubscript{2} Maecenas convallis mauris sit amet sem ultrices
gravida. Etiam eget sapien nibh. Sed ac ipsum eget enim egestas
ullamcorper nec euismod ligula. Curabitur fringilla pulvinar lectus
consectetur pellentesque. Quisque augue sem, tincidunt sit amet feugiat
eget, ullamcorper sed velit.

Sed non aliquet felis. Lorem ipsum dolor sit amet, consectetur
adipiscing elit. Mauris commodo justo ac dui pretium imperdiet. Sed
suscipit iaculis mi at feugiat. Ut neque ipsum, luctus id lacus ut,
laoreet scelerisque urna. Phasellus venenatis, tortor nec vestibulum
mattis, massa tortor interdum felis, nec pellentesque metus tortor nec
nisl. Ut ornare mauris tellus, vel dapibus arcu suscipit sed. Nam
condimentum sem eget mollis euismod. Nullam dui urna, gravida venenatis
dui et, tincidunt sodales ex. Nunc est dui, sodales sed mauris nec,
auctor sagittis leo. Aliquam tincidunt, ex in facilisis elementum,
libero lectus luctus est, non vulputate nisl augue at dolor. For more
information, see \nameref{s1-appendix}.

\hypertarget{supporting-information}{%
\section{Supporting information}\label{supporting-information}}

\paragraph*{S1 Fig.}
\label{s1-fig}
{\textbf{Bold the title sentence.}} Add descriptive text after the title
of the item (optional).

\paragraph*{S2 Fig.}
\label{s2-fig}
{\textbf{Lorem ipsum.}} Maecenas convallis mauris sit amet sem ultrices
gravida. Etiam eget sapien nibh. Sed ac ipsum eget enim egestas
ullamcorper nec euismod ligula. Curabitur fringilla pulvinar lectus
consectetur pellentesque.

\paragraph*{S1 File.}
\label{s1-file}
{\textbf{Lorem ipsum.}}

\paragraph*{S1 Video.}
\label{s1-video}
{\textbf{Lorem ipsum.}} Maecenas convallis mauris sit amet sem ultrices
gravida. Etiam eget sapien nibh. Sed ac ipsum eget enim egestas
ullamcorper nec euismod ligula. Curabitur fringilla pulvinar lectus
consectetur pellentesque.

\paragraph*{S1 Appendix.}
\label{s1-appendix}
{\textbf{Lorem ipsum.}} Maecenas convallis mauris sit amet sem ultrices
gravida. Etiam eget sapien nibh. Sed ac ipsum eget enim egestas
ullamcorper nec euismod ligula. Curabitur fringilla pulvinar lectus
consectetur pellentesque.

\paragraph*{S1 Table.}
\label{s1-table}
{\textbf{Lorem ipsum.}} Maecenas convallis mauris sit amet sem ultrices
gravida. Etiam eget sapien nibh. Sed ac ipsum eget enim egestas
ullamcorper nec euismod ligula. Curabitur fringilla pulvinar lectus
consectetur pellentesque.

\hypertarget{acknowledgments}{%
\section{Acknowledgments}\label{acknowledgments}}

Cras egestas velit mauris, eu mollis turpis pellentesque sit amet.
Interdum et malesuada fames ac ante ipsum primis in faucibus. Nam id
pretium nisi. Sed ac quam id nisi malesuada congue. Sed interdum aliquet
augue, at pellentesque quam rhoncus vitae.


\nolinenumbers
  \bibliography{bibliography.bib}

\end{document}
